\chapter{Introdução}

\section{Motivação}

Quando uma organização precisa de servidores e computadores, 
para auxiliar as suas atividades, surge a necessidade de 
instalação e configuração de sistemas operacionais, programas 
e serviços que entrarão em operação. Essas situação, 
aparentemente simples do ponto de vista de um usuário comum
que instala os programas convencionais de que precisa, se 
transforma em uma tarefa de configuração complexa e inviável 
de ser feita para organizações com um número de servidores e 
computadores muito elevado. Essa demanda por ativos 
computacionais pode variar muito dependendo do serviço 
oferecido pela organização, pode crescer dia a dia ou 
apresentar picos sob uma demanda específica, e para se 
otimizar a relação entre custo benefício, se faz necessária
a capacidade de ativar e desativar tais sistemas 
computacionais quando for necessário. 

Assim, o processo de  
instalação dos sistemas operacionais e dos aplicativos se 
torna árduo e envolve tarefas trabalhosas e repetitivas 
para os administradores. Nesse cenário, surgiu uma 
tendência de tentar criar estruturas automatizadas que 
pudessem facilitar a integração desses sistemas, englobando
todas as fazes do processo de desenvolvimento de softwares 
e sistemas. 

A partir desse momento, os administradores não 
mas ficaram responsáveis por configurar e instalar sistemas 
de softwares, e passaram a investir seu tempo no 
desenvolvimento de ferramentas que automatizem todos os 
passos do processo. Nesse contexto, uma área chamada 
DevOps \cite{loukides2012devops}, que trata  da integração 
de operação com desenvolvimento de sistemas vem se 
apresentando e se fortalecendo, a medida em que as demandas 
por estruturas de sistemas cada vez mais flexíveis e com 
menor custo vem crescendo.


\section{Objetivo}

Esse trabalho tem como objetivo o desenvolvimento e a 
comparação de estruturas de DevOps. Cada estrutura dessa 
deverá conter uma das ferramentas usadas em cada fase do 
processo de desenvolvimento e implantação de software: 
Bancos de dados, Integração contínua, Colocar em produção
( deployment), Núvem, IaaS( Infrastructure as a Service),
PaaS( Plataform as a Service), BI, Monitoring, SMC, 
Gerencia de repositórios, Configuração e Provisionamento, 
Release Managiment, Logging, Build, Testing, Conteinerization, 
Colaboration, Security.

Assim, serão desenvolvidas estruturas dessas completas e 
funcionais, e serão feitas comparações com o objetivo de 
tentar determinar um parâmetro que possa ser útil na 
determinação de qual dessas estruturas se deve usar.

\section{Justificativa}

\cite{callanandevops}


\section{Metodologia}

\section{Estrutura}