\begin{resumo}
	Este trabalho está focado na análise e comparação de três modelos  
	de estruturas DevOps encontrados na atualidade. Em uma primeira 
	abordagem, o sistema da IBM Blue Mix será estudado, analisado, e
	terá suas características catalogadas para a futura comparação.
	Em uma segunda abordagem, será criado e testado o modelo encontrado 
	no livro Caixa de Ferramentas DevOps: um guia para a construção, 
	administração e arquitetura de sistemas modernos. Esse modelo usa 
	para criar sua estrutura DevOps as seguintes ferramentas: Linux, 
	SSH, Git, Vagrant, Ansible, exemplo da instalação de wordpress em 
	uma máquina virtual, proxy reverso, Cassandra e EC2, Métricas e 
	monitoração, Análise de performance em cloud com New Relic, Docker 
	e uma última parte que fala de técnicas para se colocar software 
	em produção.  Nessa fase, os 
	resultados também serão catalogados para as comparações e análises.
	Em uma terceira fase, será abordado o modelo encontrado no livro 
	DevOps na pratica: entrega de software confiável e automatizada. 
	Esse livro aborda os seguintes temas para a construção do seu 
	ambiente DevOps: introdução às idéias pregadas pelas técnicas 
	DevOps, uma parte falando sobre a fase de produção, uma parte 
	falando sobre monitoramento, uma parte falando sobre 
	infraestrutura como código, uma parte falando sobre Puppet, 
	uma parte falando sobre integração contínua, Pipeline de entrega 
	e uma parte que fala de tópicos avançados. Após essas implementações 
	e catalogações, será feita uma comparação dos três modelos análisando
	suas características e elencando pontos positivos e negativos.
	
	\vspace{\onelineskip}
	\noindent
	\textbf{Palavras-chave}: DevOps, desenvolvimento, operação, ambientes.
\end{resumo}

\begin{resumo}[Abstract]
	\begin{otherlanguage*}{english}		
This work is focused on analysis and comparison of three 
DevOps models of structures found today. In a first
approach, the IBM Blue Mix system will be studied, analyzed, and
will have its characteristics cataloged for future comparison.
In a second approach, the model found
in the book Caixa de Ferramentas DevOps - um guia para a construção, 
	administração e arquitetura de sistemas modernos will be created
	and tested. This model uses
to create its DevOps structure the following tools: Linux,
SSH, Git, Vagrant, Ansible, wordpress installation example
a virtual machine, reverse proxy, Cassandra and EC2, and Metrics
monitoring, cloud-performance analysis with New Relic, Docker
and a last part that talks about techniques to put software
	in production. In this phase,
results will also be cataloged for comparisons and analysis.
In a third phase, the model found in the book
DevOps na pratica - entrega de software confiável e automatizada will 
be created and tested.
This book covers the following topics for the construction of its
DevOps environment: introduction to the ideas preached by technical
DevOps, a part talking about the production phase, a part
talking about monitoring, a part talking about
infrastructure as code, a part talking about Puppet,
a part talking about continuous integration, delivery Pipeline
and a part that speaks of advanced topics. After these implementations
and catalogations, the three models will be compared and analysed and 
the positives and negatives points will be pointed.

		\vspace{\onelineskip}
		\noindent
		\textbf{Keywords}: DevOps, development, operation, environment.
	\end{otherlanguage*}
\end{resumo}