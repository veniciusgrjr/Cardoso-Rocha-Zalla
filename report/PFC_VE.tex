% !TEX encoding = UTF-8 Unicode
\documentclass[12pt,openany, oneside, a4paper,brazil]{abntex2}
\usepackage{indentfirst}
\usepackage[utf8]{inputenc}
\usepackage[T1]{fontenc}
\usepackage{lmodern} %fonte
\usepackage{lastpage} %ficha catalografica
\usepackage[num, overcite]{abntex2cite}
\usepackage{amsmath}
\usepackage{float}

%%%%%%%%%%%%%%%%%%%%%%%%%%%%%%%%%%%%%%%%%%%%%%%%%%%% Exibir códigos em c++
\usepackage{xcolor}
% Definindo novas cores
\definecolor{verde}{rgb}{0,0.5,0}
% Configurando layout para mostrar codigos C++
\usepackage{listings}
\lstset{
  language=C++,
  basicstyle=\ttfamily\small, 
  keywordstyle=\color{blue}, 
  stringstyle=\color{verde}, 
  commentstyle=\color{red}, 
  extendedchars=true, 
  showspaces=false, 
  showstringspaces=false, 
  numbers=left,
  numberstyle=\tiny,
  breaklines=true, 
  backgroundcolor=\color{green!10},
  breakautoindent=true, 
  captionpos=b,
  xleftmargin=0pt,
}

%%%%%%%%%%%%%%%%%%%%%%%%%%%%%%%%%%%%%%%%%%%%%%%%%%%%%%% Fim Exibir códigos em c++

\citebrackets[]
\setlength{\parskip}{0.75cm} % tente também \onelineskip

\titulo{DevOps: aproximando a área de desenvolvimento da operacional}
\autor{Luan Ferreira Cardoso \and Ricardo Sollon Zalla \and Venicius Gonçalves da Rocha Junior}
\data{\today}
\instituicao{Instituto Militar de Engenharia}
\local{Rio de Janeiro}
\tipotrabalho{Projeto de Fim de Curso}
\orientador{Clayton Escouper das Chagas}
\coorientador{Coorientador ?????}
\preambulo{Trabalho apresentado ao Curso de Engenharia de Computação do Instituto Militar de Engenharia como Verificação Especial do Projeto de Fim de Curso.}

\renewcommand{\imprimircapa}{
	\begin{capa}
%		\begin{figure}[t]
%		    \centering
%		    \includegraphics{brasao}
%		    \caption{Roadmap da experiência de DevOps no Bluemix}
%		    \label{Rotulo}
%		\end{figure}
		\center
		\large{\textbf{ MINISTÉRIO DA DEFESA\\
			EXÉRCITO BRASILEIRO\\
			DEPARTAMENTO DE CIÊNCIA E TECNOLOGIA\\
			INSTITUTO MILITAR DE ENGENHARIA\\
			ENGENHARIA DE COMPUTAÇÃO - SE/8\\
		}}
		\vspace{5cm}
		\imprimirautor\\
		\vspace{5cm}
		\large{\textbf{\imprimirtitulo}}\\
		\vspace{5cm}
		\imprimirlocal\\
		\imprimirdata

	\end{capa}
}


\begin{document}
\imprimircapa
\imprimirfolhaderosto{}
\begin{fichacatalografica}
	\begin{tabular}{@{\hspace{0cm}}p{14.8cm}}
	c2016\\
	\\
	\imprimirinstituicao\\
	Praça General Tibúrcio, 80 - Praia Vermelha\\
	Rio de Janeiro - RJ \hspace{1.5cm} CEP: 22290-270\\
	\\
	Este exemplar é de propriedade do \imprimirinstituicao, que poderá incluí-lo em base de dados, armazenar em computador, microfilmar ou adotar qualquer forma de arquivamento.\\
	\\
	É permitida a menção, reprodução parcial ou integral e a transmissão entre bibliotecas deste trabalho, sem modificação de seu texto, em qualquer meio que esteja ou venha a ser fixado, para pesquisa acadêmica, comentários e citações, desde que sem finalidade comercial e que seja feita a referência bibliográfica completa.\\
	\\
	Os conceitos expressos neste trabalho são de responsabilidade dos autores e do orientador.\\
	\\
	\\
	\\
	\end{tabular}

	\small
	\begin{center}
	\begin{tabular}{|cp{13cm}|} \hline
		\hspace{1.3cm} & \\
		& Cardoso, Luan; Zalla, Ricardo e Gonçalves, Venicius \\
		\hspace{0.2cm} S586d & \hspace{0.3cm} \imprimirtitulo{} / \imprimirautor. - \imprimirlocal: \imprimirinstituicao, 2016. \\
		& \\
		& \hspace{0.65cm} \pageref{LastPage}f. : il., graf., tab. : -cm. \\
		& \\
		& \hspace{0.6cm} \imprimirtipotrabalho{} - \imprimirinstituicao \\
		& \hspace{0.6cm} \imprimirorientadorRotulo{} \imprimirorientador.\\
		& \\
		& \hspace{0.6cm} 1 - DevOps \hspace{0.1cm} 2 - Desenvolvimento e Operação\\
		
		& \\ 
		& \hspace{9.75cm} CDU ???.???.?? \\
		\hline
	\end{tabular}
	\end{center}
\end{fichacatalografica}
\begin{folhadeaprovacao}
	\begin{center}
		{\ABNTEXchapterfont\large\imprimirautor}

		\vspace*{\fill}\vspace*{\fill}
		\begin{center}
			\ABNTEXchapterfont\bfseries\Large\imprimirtitulo
		\end{center}
		\vspace*{\fill}
		
		\hspace{.45\textwidth}
			\begin{minipage}{.5\textwidth}
		    	\imprimirpreambulo
		    \end{minipage}
		\vspace*{\fill}
	\end{center}

	Trabalho aprovado. \imprimirlocal, \imprimirdata:

	\assinatura{\textbf{Prof. \imprimirorientador} \\ Orientador, D. Sc., do IME}
	\assinatura{\textbf{Prof. Anderson Fernandes Pereira dos Santos} \\ Convidado, D. Sc., do IME}
	\assinatura{\textbf{Prof. Julio Cesar Duarte} \\ Convidado, D. Sc., do IME}

	\begin{center}
		\vspace*{0.5cm}
		{\large\imprimirlocal}
		\par
		{\large\imprimirdata}
		\vspace*{1cm}
	\end{center}
\end{folhadeaprovacao}

\tableofcontents*
\newpage

% \addcontentsline{toc}{chapter}{\listfigurename}
% \listoffigures*
% \cleardoublepage

% \addcontentsline{toc}{chapter}{\listtablename}
% \listoftables*
% \cleardoublepage

% \addcontentsline{toc}{chapter}{\listadesiglasname}
% \begin{siglas}
	\item[Fig.] Figura
\end{siglas}

% \addcontentsline{toc}{chapter}{\listadesimbolosname}
% \begin{simbolos}
	\item[$ \Gamma $] Letra grega Gamas
\end{simbolos}

\begin{resumo}
	Resumo em pt

	\vspace{\onelineskip}
	\noindent
	\textbf{Palavras-chave}: DevOps, desenvolvimento, operação, ambientes.
\end{resumo}

\begin{resumo}[Abstract]
	\begin{otherlanguage*}{english}
		Abstract in English

		\vspace{\onelineskip}
		\noindent
		\textbf{Keywords}: DevOps, development, operation, environment.
	\end{otherlanguage*}
\end{resumo}

\textual

\chapter{Introdução}

\section{Motivação}

Quando uma organização precisa de servidores e computadores, 
ou precisa desenvolver um software e liberá-lo para os usuários, 
 ou ainda precisa de mais claboração e cominicação entre as equipes 
 devido a peculiaridades de alguns projetos, surge a necessidade de 
instalação e configuração de sistemas operacionais, programas 
e serviços que entrarão em operação ao final do projeto. Essas situação, 
aparentemente simples do ponto de vista de um usuário comum
que instala os programas convencionais de que precisa, se 
transforma em uma tarefa de configuração complexa e inviável 
de ser feita para organizações com um número de servidores e 
computadores muito elevado \cite{humble2011enterprises}. 
Essa demanda por ativos 
computacionais pode variar muito dependendo do serviço 
oferecido pela organização, pode crescer dia a dia ou 
apresentar picos sob uma demanda específica, e para se 
otimizar a relação entre custo benefício, e para possibilitar 
uma entrega contínua e confiável \cite{humble2010continuous},
se faz necessária a capacidade de automatizar o processo de 
desenvolvimento e implantação de tais sistemas 
computacionais quando for necessário. 

Assim, o processo de  
instalação dos sistemas operacionais e dos aplicativos se 
torna árduo e envolve tarefas trabalhosas e repetitivas 
para os administradores e desenvolvedores \cite{httermann2012devops}. 
Nesse cenário, surgiu uma tendência de tentar criar 
estruturas automatizadas que pudessem facilitar a 
integração desses processos de desenvolvimento de sistemas 
\cite{humble2011enterprises}, englobando todas as 
fazes do processo de desenvolvimento de softwares 
e sistemas. 

A partir desse momento, os administradores não 
mas ficaram responsáveis por configurar e instalar sistemas 
de softwares, e passaram a investir seu tempo no 
desenvolvimento de ferramentas que automatizem todos os 
passos do processo. Nesse contexto, uma área chamada 
DevOps \cite{loukides2012devops}, que trata  da integração 
de operação com desenvolvimento de sistemas vem se 
apresentando e se fortalecendo, a medida em que as demandas 
por estruturas de sistemas cada vez mais flexíveis e com 
menor custo vem crescendo.


\section{Objetivo}

Com o constante crescimento da comunidade DevOps, o suporte 
e o número de ferramentas e alternativas disponíveis estão 
aumentando constantemente. Assim, já é possível encontrar 
diversas ferramentas de DevOps, incluindo artigos, scripts 
e softwares, que podem ser reusadas para 
automatizar o processo de colocar um software em produção 
como é possível observar em \cite{nelson2013test} e 
\cite{sabharwal2014automation}. 

Esse trabalho tem como objetivo o desenvolvimento e a 
comparação de estruturas de DevOps. Cada estrutura dessa 
deverá conter uma das ferramentas usadas em cada fase do 
processo de desenvolvimento e implantação de software: 
Bancos de dados, Integração contínua, Colocar em produção
( deployment), Núvem, IaaS( Infrastructure as a Service),
PaaS( Plataform as a Service), BI, Monitoring, SMC, 
Gerencia de repositórios, Configuração e Provisionamento, 
Release Managiment, Logging, Build, Testing, Conteinerization, 
Colaboration, Security.

Assim, serão desenvolvidas estruturas dessas completas e 
funcionais, e serão feitas comparações com o objetivo de 
tentar determinar um parâmetro que possa ser útil na 
determinação de qual dessas estruturas se deve usar.

\section{Justificativa}

Para mostrar a importância desse trabalho, é possível citar 
alguns casos de sucesso da implementação da metodologia DevOps 
e analisar as melhorias que essa nova abordagem trouxe 
para essas organização.

Inicialmente, pode-se citar o grupo empresarial WOTIF GROUP que 
atua no comércio de viagens com uma plataforma na internet, 
segundo \cite{callanandevops}. Em 2013 e 2014, a organização 
reorganizou o seu processo de liberação de softwares, reduzindo 
o tempo médio de liberação de software de semanas para horas, 
ratificando a importância dessa nova metodologia. Em resumo, 
uma das principais dificuldades encontradas pela empresa era 
que seus diversos departamentos de engenharia queriam colaborar 
nas fases de desenvolvimento de infraestrutura, de teste e de 
colocar em produção, mas não conseguiam encontrar uma maneira 
de fazer isso. Assim essa organização conseguiu resolver seus 
problemas utilizando as técnicas de DevOps e criando uma cadeia 
de ferramentas que atendeu às suas expectativas.


%\chapter{Ferramentas DevOps}

%

    \section{Bancos de dados}

	\subsection{Oracle}
	\subsection{MySQL}
	\subsection{MSSQL}


    \section{Integração contínua}

	\subsection{Jenkins}
	\subsection{Bamboo}
	\subsection{Travis CI}
	\subsection{Codeship}

    \section{Deployment}

	\subsection{Ssh}
	\subsection{Deployment Manager}
	\subsection{SmartFrog}
	\subsection{Capistrano}

    \section{Núvem, IaaS( Infrastructure as a Service), PaaS( Plataform as a Service)}

	\subsection{Amazon AWS}
	\subsection{Azure}
	\subsection{Heroku}
	\subsection{Rachspace}

    \section{Monitoramento}

	\subsection{Kibana}
	\subsection{New Relic}
	\subsection{Nagios}
	\subsection{Ganglia}

    \section{SMC}

	\subsection{Git}
	\subsection{Subversion}
	\subsection{Github}
	\subsection{Bitbucket}
	
	
    \section{Gerencia de repositórios}

	\subsection{Archiva}
	\subsection{Nexus}
	\subsection{Artifactory}
	\subsection{NuGet}
	
    \section{Configuração e provisionamento}

	\subsection{Chef}
	\subsection{Puppet}
	\subsection{Ansible}
	\subsection{Salt}
	\subsection{BladeLogic}
	\subsection{Vagrant}
	\subsection{TerraForm}
	\subsection{Cobbler}
	\subsection{Bcfg2}
	\subsection{CFEngine}
	
    \section{Release Managiment}

	\subsection{XL Release}
	\subsection{UrbanCodeRelease}

    \section{Logging}
    \section{Build}
    \section{Testing}
    \section{Conteinerization}
    \section{Colaboration}
    \section{Security}
    
	
	
	
	
	
	
	
	
	
	
	
	
	



%\chapter{Estruturas de devops completas}

\section{Metodologia}

Inicialmente foi realizado um levantamento bibliogr�fico a procura de boas refer�ncias e ferramentas estabelecidas no contexto da metodologia DevOps. A metodologia DevOps � uma rea��o � interdepend�ncia entre desenvolvimento de software e opera��es de TI. Pretende ajudar organiza��es a produzir software e servi�os rapidamente. Desde que associa��es de profissionais e blogs est�o tratando do tema somente desde 2009 n�o existem bibliografias cl�ssicas ou ferramentas estabelecidas. Desse modo optamos pela montagem de tr�s plataformas DevOps: duas plataformas customizadas com ferramentas de c�digo aberto e uma arquitetura propriet�ria. O intuito � que no fim do trabalho possamos estabelecer m�tricas a fim de comparar o desempenho das tr�s plataformas.
O primeiro passo para iniciar a montagem das plataformas � a configura��o do ambiente onde rodar�o as ferramentas das plataformas.
Como o foco do trabalho n�o � no desenvolvimento da aplica��o em si, mas sim nas plataformas de integra��o entre o time de desenvolvimento e o time de opera��es utilizaremos aplica��es t�o somente para estabelecimento de m�tricas de usabilidade entre as plataformas. Sendo assim, ap�s a configura��o do ambiente o seguinte passo ser� a realiza��o do Build do projeto da aplica��o.
Com o Build realizado, o pr�ximo passo � o estabelecimento de testes automatizados de forma que se possa garantir que a aplica��o/funcionalidade que ir� para a produ��o conta com um alto grau de confiabilidade.
No pr�ximo passo, o Deploy, contamos finalmente com a aplica��o em produ��o. Restam-se agora o estabelecimento das m�tricas de monitora��o e a an�lise entre as arquiteturas.

\chapter{Conclusões}
Texto Conclusão

\postextual
%%\bibliographystyle{abbrv}
\bibliography{refs}
\end{document}