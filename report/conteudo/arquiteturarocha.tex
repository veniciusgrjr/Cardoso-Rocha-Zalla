%%%%%%%%%%%%%%%%%%%%%%%%%%%%%%%%%%%%%%%% Exibir código fonte
%\begin{lstlisting}
%\end{lstlisting}

%%%%%%%%%%%%%%%%%%%%%%%%%%%%%%%%%%%%%%%% Lista com pontos
% \begin{itemize}
%   \item item1
%   \item item2
% \end{itemize}

%%%%%%%%%%%%%%%%%%%%%%%%%%%%%%%%%%%%%%%% Lista com números
%  \begin{enumerate}
%   \item item1
%   \item item2
% \end{enumerate}
 

\section{Segunda Arquitetura Customizada }

\subsection{Criação de chave pública SSH}

Primeiramente, essa arquitetura se inicia com a 
configuração de chaves públicas. Na maioria dos 
provedores, será encontrada uma opção para adicionar 
uma chave SSH pública.
Para criar uma chave RSA local seguiu-se os 
seguintes passos:
 \begin{enumerate}
   \item Digite o comando a seguir:
	
      \begin{lstlisting}
ssh-keygen -t rsa
      \end{lstlisting}
   
   \item Quando for perguntado sobre uma frase digite 
      enter duas vezes.
    \item Serão criados dois arquivos:
    \begin{lstlisting}
id_rsa_devops e id_rsa_devops.pub
    \end{lstlisting}
    
    \item Os arquivos com extenção .pub são os únicos 
	  que devem ser copiados para outras máquinas 
	  e representam as chaves públicas.

 \end{enumerate} 
 
\subsection{Configuração do controle de versão com Git e GitHub}
Para gerenciar o versionamento, foi escolhida a ferramenta Git e 
seguiu-se os seguintes paços para realizar sua instalação:

  \begin{enumerate}
   \item Para linux baseado em debian:
      \begin{lstlisting}
sudo apt-get install git-core
      \end{lstlisting}
      
   \item Para linux CentOS/RH
      \begin{lstlisting}
sudo yum install git
      \end{lstlisting}
   
   \item Para MacOSX com homebrew
      \begin{lstlisting}
brew install git
      \end{lstlisting}
 \end{enumerate}
Após a instalação do Git, foram executados os seguintes 
comandos para configurar um nome de usuário e um email, 
que serão usados toda vez em que for necessário se 
conectar ao repositório remoto: 
      \begin{lstlisting}
git config --global user.name "Seu nome"
git config --global user.email "seu@email.com.br"
      \end{lstlisting}
Nesse momento o ambiente está com o Git instalado 
e com nome e e-mail configurados. Será adotado 
agora um sistema para controlar remotamente os repositórios 
usados para construir o modelo de DevOps proposto.

O “Github” é um site que fornece um serviço de 
controle de repositórios remotos de “Git” sem custo,
caso o projeto seja aberto, e com custo caso o projeto 
seja privados. Para utiliza-lo de forma 
gratuita, foi acessado o site “github.com”, e foi criada 
uma conta. Neste ponto, será necessária a chave “ssh” que 
foi criada anteriormente, que está em:
      \begin{lstlisting}
~/.ssh/id_rsa.pub.
      \end{lstlisting}
A partir desse ponto, foi criado um novo repositório 
que posteriormente 
será sincronizado com o repositório local que estará em 
cada máquina que estiver trabalhando no projeto. Para criar 
esse repositório, foram  seguidos os paços do site clicando 
no botão de novo repositório. Para esse projeto, será dado 
o nome de projeto-simples. Assim, será criado um diretório 
local e, posteriormente, será feita a associação desse 
diretório local com o repositório remoto criado 
anteriormente no GitHub. Para isso, seguiu-se os 
passos listados abaixo:
\begin{enumerate}
   \item Para criar o diretório local foi digitado na linha de comando:
      \begin{lstlisting}
mkdir projeto-simples
      \end{lstlisting}
      
   \item Entrou-se nesse diretório com o comando:
      \begin{lstlisting}
cd projeto-simples
      \end{lstlisting}
   
   \item Com um editor de textos, foi criado um arquivo 
   de exemplo, com nome de README.md, para poder ser sincronizado 
   com o repositório remoto criado, explicando a função do projeto. 
   Foi colocado o seguinte texto no arquivo:
   
      \begin{lstlisting}
# README do meu projeto-simples
      \end{lstlisting}
   
   \item Esse projeto será apenas um shell script que conta 
   itens únicos no diretório \/etc. Assim, será criaco o 
   seguinte script com o nome de itens\_unicos.sh:
   
      \begin{lstlisting}
#!/bin/sh
Echo "Itens unicos"
Ls /etc | cut -d" -f 1 | sort | uniq | wc -l
      \end{lstlisting} 
 \end{enumerate}
Nesse momento, já foi criado o projeto simples que será 
sincronizado com o repositório remoto no github. Foram  
seguidos os passos a seguir para realizar essa sincronização:
\begin{enumerate}
   \item Para iniciar o repositório Git no diretório local:
      \begin{lstlisting}
git init
      \end{lstlisting}
      
   \item Para adicionar todos os arquivos modificados ao 
   conjunto de modificações que serão enviadas para serem 
   sincronizadas com o repositório remoto: 
      \begin{lstlisting}
git add . 
      \end{lstlisting}
   
   \item Para criar uma mensagem, descrevendo as modificações: 
      \begin{lstlisting}
git commit -m "mensagem descrevendo a alteração"
      \end{lstlisting}
      
   \item Para vincular o repositório local ao repositório remoto: 
      \begin{lstlisting}
git remote add origin git@github.com:veniciusgrjr/projeto-simples.git
      \end{lstlisting}
      
    \item Para enviar as alterações: 
      \begin{lstlisting}
git push -u origin master
      \end{lstlisting}
 \end{enumerate}
Após isso, a página do repositório remoto foi atualizada e os 
arquivos locais puderam ser visializados remotamente. Foi feito 
agora um pequeno resumo dos comandos do Git:
\begin{itemize}
    \item Esse comando Inicia um repositório Git no 
    diretório atual.
	\begin{lstlisting}
Git init
	\end{lstlisting}
    \item Adiciona um ou mais arquivos para serem 
    enviados (commit) ao repositório.
	\begin{lstlisting}
Git add.
	\end{lstlisting}
    \item Confirma as mudanças e 
    cria um commit com uma mensaApós isso, faça reload da 
    página do repositório, e os arquivos locais estarão lá.
	\begin{lstlisting}
Git commit -m "mensagem"
	\end{lstlisting}    
    \item Esse comando adiciona um “remote” 
    ao repositório atual, chamado origin. Você poderia trabalhar 
    sem ter um remote, não é mandatório.
	\begin{lstlisting}
Git remote ...
	\end{lstlisting}
    \item Envia (push) as modificações 
    para o repositório remoto. O parametro -u só é necessário na 
    primeira execução.
	\begin{lstlisting}
Git push -u origin master
	\end{lstlisting}
    \item Cria uma cópia o repositório dado pela URL 
    para a máquina local em que foi digitado.
	\begin{lstlisting}
Git clone
	\end{lstlisting}
    \item Mostra o estado atual do repositório e 
    das mudanças.
	\begin{lstlisting}
Git status
	\end{lstlisting}
\end{itemize}
Agora, para testar a ferramenta de sincronização, o diretório 
local será modificado e sincronizado com o repositório no Github. 
Para isso, foi criado um arquivo com o nome de portas.sh, 
com o seguinte código:
    \begin{lstlisting}
#!/bin/sh
echo "Lista de porta 80 no netstat"
netstat -an | grep 80
    \end{lstlisting}
 Para adicionar essa modificação ao repositório no Github, 
 foram executados os seguintes comandos:
     \begin{lstlisting}
Git add portas.sh
Git commit -m "add portas.sh"
Git push origin master
    \end{lstlisting}
\subsection{Instalação e configuração do Vagrant e do Virtualbox}

Uma máquina virtual parada é uma imagem de um disco 
de metadados que descrevem sua configuração: processador, 
memória, discos e conexões externas. A mesma máquina em 
execução é um processo que depende de um scheduler( agendador 
de processos) para coordenar o uso dos recursos locais. Para 
gerenciar máquinas virtuais, pode-se usar as interfaces das 
aplicações VirtualBox, Parallels ou VMW, ou pode-se utilizar 
bibliotecas e sistemas que abstraem as diferenças entre 
essas plataformas, com interface consistente para criar, 
executar, parar e modificar uma máquina virtual.
Para criar e gerenciar os ambientes de máquinas virtuais 
locais, foi escolhido o software Vagrant (www.vagrantup.com). 
Para executar as máquinas virtuais, foi escolhido o software 
VirtualBox(www.virtualbox.org).

O Vagrant gerencia e abstrai provedores de máquinas 
virtuais locais e públicos( VMWare, VirtualBox, Amazon 
AWS, DigitalOcean, entre outros). Ele tem uma linguagem 
especifica (DSL) que descreve o ambiente e suas máquinas. 
Além disso, ele fornece interface para os sistemas de 
gerenciamento de configuração mais comuns como Chef, 
Puppet, CFEngine e Ansible. Ele é um software que cria e 
configura ambientes 
virtuais de desenvolvimento. Possui interface de 
linha de comando simples para subir e interagir 
com esses ambientes virtuais. Essa ferramenta ajuda na 
criação da infraestrutura para o projeto, usando para 
isso uma máquina virtual. Nesse momento surge um questionamento: 
será preciso uma máquina virtual para cada projeto, isso não 
complicaria ainda mais o projeto? A resposta é não. O Vagrant 
deixa muita coisa invísivel, possibilitando se preocupar apenas 
com o código. Funciona como uma máquina virtual reduzida e 
portável. Para cada projeto é possível deixar 
um ambiente rodando PHP 4, outro PHP 5, outro Debian e outro 
CentOS.

Para instalação do Virtualbox, foram seguidos os paços do site  
www.virtualbox.org. Após isso, para testar se tudo está 
funcionando, foi feito o download de uma ISO do ubuntu 
em www.ubuntu.com/download/server e testou-se a criação 
de máquinas virtuais.

Para instalar o Vagrant, foi acessado www.vagrantup.com e, 
foram seguidos os passos da instalação. Para criar uma 
máquina virtual usando o Vagrant seguiu-se os paços
abaixo:
\begin{enumerate}
   \item Foi criado um diretório chamado testvm. 
   \item Foi criado um arquivo chamado vagrantfile e 
   digitado nele:
      \begin{lstlisting}
# -*- mode: ruby -*-
# vi: set ft=ruby :

VAGRANTFILE_API_VERSION = “2”

Vagrant.configure( VAGRANTFILE_API_VERSION ) do |config|

	config.vm.define “testvm” do |testvm|
		testvm.vm.box = “ubuntu/trusty64”
		testvm.vm.network : private_network, ip : “192.168.33.21”
	end
	config.vm.provider “virtualbox” do |v|
		v.customize [“modifyvm”, :id, “--memory”, “1024”]
	end
end
      \end{lstlisting}
  \item No diretório desse arquivo criado,para subir o 
  servidor digitou-se:
    \begin{lstlisting}
vagrant up
    \end{lstlisting}

      \item Para testar, foi feita uma conexão com a 
      máquina virtual criada digitando-se:
    \begin{lstlisting}
vagrant ssh
    \end{lstlisting}
   \item É importante notar que dentro da máquina 
    o seu diretório local foi mapeado como um mount 
   point dentro da máquina virtual, acessando o 
   diretório vagrant digitando cd /vagrant é possível 
   observar isso. É possível criar este ponto com 
   outro nome ou deixar de criá-lo de acordo com a 
   configuração do Vagrantfile.
   \item Para destruir a máquina virtual criada, 
   basta digitar no terminal: 
    \begin{lstlisting}
vagrant destroy
//seguido de y, quando for perguntado se pode realmente destruir a máquina virtual
    \end{lstlisting}
\end{enumerate}
Assim, resumindo os comandos, temos:
\begin{enumerate}
   \item Para subir a máquina virtual:
    \begin{lstlisting}
vagrant up
    \end{lstlisting}
   \item Para se conectar à máquina virtual:
    \begin{lstlisting}
vagrant ssh
    \end{lstlisting}   
   \item Para destruir a máquina virtual:
    \begin{lstlisting}
vagrant destroy
    \end{lstlisting}
   \item Para desativar a máquina virtual:
    \begin{lstlisting}
vagrant halt
    \end{lstlisting}    
   \item Para executar somente o provisionamento:
    \begin{lstlisting}
vagrant provision
    \end{lstlisting}
\end{enumerate}
Usar os processos fornecidos pelo VirtualBox, 
por exemplo, para criar máquinas virtuais se torna um 
processo demorado e difícil de ser replicado. Usando 
esse Vagrantfile, é possível subir( up) a mesma máquina 
várias vezes, em ocasiões distintas sem usar a 
interface do VirtualBox. É possível versionar este 
arquivo que descreve a  máquina virtual junto com seu 
código e quando mudar a configuração de memória, por 
exemplo, esta mudança estará no histórico junto às 
mudanças de código.

O Vagrant e sua configuração utiliza Ruby. Existem 
versões distintas de APIs e nesse trabalho será utilizada 
a versão 2. Assim, dentro de uma instância da configuração 
definiu-se máquinas e suas características, em um bloco de 
código Ruby.
O Vagrant implementa o conceito de provisionador com 
drivers para quase todos os sistemas de gerenciamento de 
configuração existentes. Estes sistemas vão desde receitas 
simples para instalar pacotes até agentes que verificam a 
integridade de arquivos de configuração e variáveis do 
sistema. Vamos utilizar um driver de provisionamento do 
vagrant que permite que um arquivo com comandos do shell 
seja executado logo após a criação da máquina virtual. 
Assim será criado o arquivo webserver.sh no mesmo diretório 
e com o seguinte código: 
\begin{lstlisting}
#!/bin/bash

echo “Atualizando repositorio”
sudo apt-get update
echo “Instalando o nginx”
sudo apt-get -y install nginx
\end{lstlisting}
Agora o vagrantfile será editado para que o provisionamento 
seja ativado. ele deve ficar com o exposto abaixo:
\begin{lstlisting}
# -*- mode: ruby -*-
# vi: set ft=ruby :

VAGRANTFILE_API_VERSION = “2”

Vagrant.configure( VAGRANTFILE_API_VERSION ) do |config|

	config.vm.define “testvm” do |testvm|
		testvm.vm.box = “ubuntu/trusty64”
		testvm.vm.network : private_network, ip : “192.168.33.21”
		testvm.vm.provision “shell”, path: “webserver.sh”
	end
	config.vm.provider “virtualbox” do |v|
		v.customize [“modifyvm”, :id, “--memory”, “1024”]
	end
end
\end{lstlisting}
Agora a máquina que está rodando será 
destruída com vagrant destroy e criada novamente com 
vagrant up.Observando a saída do terminal, e fazendo 
o teste no browser, percebe-se que o nginx foi 
instalado com sucesso sem a necessidade de se conectar 
por ssh para executar o comando que instala o nginx na 
máquina virtual.

Vamos instalar, agora, o PHP5 na máquina virtual. Para isso, 
será modificado o arquivo webserver.sh de acordo com o código
a seguir:
\begin{lstlisting}
#!/bin/bash

echo “Atualizando repositorio”
sudo apt-get update
echo “Instalando o nginx”
sudo apt-get -y install nginx
echo “instalando PHP”
sudo apt-get install -y php5-fpm
\end{lstlisting}
Nesse ponto, não foi preciso destruir e recriar a máquina para 
executar novamente o provisionamento. Foi aproveitado o fato 
de que o gerenciador de pacotes não instalará um pacote duas 
vezes. Assim, é possível executar o mesmo script, que só o PHP 
seria instalado. O Vagrant oferece um comando que ativa somente 
a fase do provisionamento, vagrant provision. Assim Executando 
vagrant provision no terminal, observa-se a execução do script. 
Note que os comandos do Vagrant só funcionaram dentro 
do diretório em que o vagrantfile está. 
Com o comando ls -larth é possível ver que foi criado um diretório 
chamado .vagrant que contém todos os dados do ciclo de vida e 
provisionamento da máquina virtual.

\subsection{Instalação e configuração do Ansible}

É um sistema de automação de configuração feito em python, 
que permite descrever procedimentos em arquivos no formato 
YAML que são reproduzidos utilizando SSH em máquinas remotas. 
Existem outras ferramentas desta categoria que executam 
localmente e que fornecem sevidores para o gerenciamento de 
dados remotamente, como CHEF, Puppet e CFEngine. Para essa 
arquitetura será utilizado  
o Ansible localmente sem seu servidor de dados, o Ansible Tower.
O Ansible vem com bibliotecas completas para quase todas as 
tarefas, além de uma linguagem de template. Além das tarefas 
dentro de um servidor, o Ansible fornece módulos para o 
provisionamento de máquinas e aplicações remotas. 
Provisionar é o jargão para instalar e configurar itens de 
infraestrutura e plataforma como máquinas, bancos de dados 
e balanceadores de carga.

Sistemas como o Ansible implementam tarefas com uma 
característica importante: idempotência. A idempotência é 
uma propriedade que, aplicada ao gerenciamento de configuração, 
garante	que as operações terão o mesmo resultado independentemente 
do momento em que serão aplicadas. A criação de uma máquina 
utilizando este conjunto de configurações sempre terá o 
resultado previsível.

Para instalar no ubuntu, utilizou-se os seguintes comandos:
\begin{lstlisting}
sudo apt-get install software-properties-common
sudo apt-add-repository ppa:ansible/ansible
sudo apt-get update
sudo apt-get install ansible
\end{lstlisting}
Para testar se foi instalado corretamente, foi digitado no terminal 
ansible-playbook -h, para verificar se aparece a ajuda do Ansible.

Para configurar o Ansible, foi criado no diretório testvm o 
arquivo com o código a seguir, com o nome de webserver.yml:
\begin{lstlisting}
- hosts: all
  sudo: True
  user: vagrant
  tasks:
    - name: "Atualiza pacotes"
      shell: sudo apt-get update
    - name: "Instala o nginx"
      shell: sudo apt-get -y install nginx
\end{lstlisting}
Esse formato de arquivo funciona como um dicionário no 
formato: chave: valor. Agora, temos que reconfigurar o Vagrant,
para isso o arquivo vagrantfile deve ficar do modo a seguir:
\begin{lstlisting}
# -*- mode: ruby -*-
# vi: set ft=ruby :

VAGRANTFILE_API_VERSION = “2”

Vagrant.configure( VAGRANTFILE_API_VERSION ) do |config|

	config.vm.define “testvm” do |testvm|
		testvm.vm.box = “ubuntu/trusty64”
		testvm.vm.network : private_network, ip : “192.168.33.21”
		testvm.vm.provision “ansible” do |ansible|
			ansible.playbook = “webserver.yml”
			ansible.verbose = “vvv”
		end
	end
	config.vm.provider “virtualbox” do |v|
		v.customize [“modifyvm”, :id, “--memory”, “1024”]
	end
end
\end{lstlisting}
A definição de máquina virtual de testvm tem um atributo 
que define um provisionador. Na configuração anterior, foi 
utilizado um shell script para instalar o nginx. Com essa 
mudança, o Vagrant utilizará o Ansible como provisionador 
com o playbook webserver.yml. Usou-se o atributo verbose 
do Ansible com o valor vvv para indicar que todas as 
mensagens devem ser enviadas para o console, e assim 
pode-se observar os comandos sendo executados.

Toda task é uma função de um módulo de Ansible. 
Os módulos que vêm na instalação padrão são chamados de 
Core Modules. É possível construir módulos utilizando a 
linguagem Python e estender o Ansible. Agora o playbook 
precisa ser refatorado, para utilizar um módulo core chamado 
apt\_module (http://docs.ansible.com/apt\_module.html ) 
em vez do módulo shell. O módulo shell é útil para 
automatizar muitas tarefas, mas o Ansible possui módulos 
especializados que cuidam da consistência da tarefa e 
criam estados para manter a idempotência. Assim, o 
arquivo webserver.yml deve ficar desse modo:
\begin{lstlisting}
- hosts: all
  sudo: True
  user: vagrant
  tasks:
    - name: "Atualiza pacotes e instala nginx"
      apt: name=nginx state=latest update_cache=yes
      install_recommends=yes
\end{lstlisting}
Note que duas tarefas foram reduzidas a uma que utiliza 
o módulo apt e diz para o Ansible instalar o nginx na 
última versão presente no repositório. Um dos atributos 
desta tarefa está indicando que um update no cache do 
gerenciador de pacotes deve ser executado. Outro atributo 
indica que as dependências recomendadas devem ser instaladas 
automaticamente. Note que, utilizando o módulo shell, foi 
preciso pedir explicitamente pela atualização do 
repositório e também adicionar o parâmetro -y ao comando 
apt-get install para evitar que a task ficasse esperando 
uma entrada do usuário.

O Ansible também fornece um módulo para o gerenciador 
de pacotes yum (http://docs.ansible.com/yum\_module.html) 
e diretivas condicionais que podem ser usadas para 
detectar o sistema operacional e distribuições de Linux. 
Veja como declarar a mesma task para instalar nginx para 
duas famílias de distribuições de Linux utilizando 
o condicional “when” e variáveis internas do Ansible:
\begin{lstlisting}
- name: Install	the nginx packages
  yum: name={{ item }} state=present
  with_items: redhat_pkg
  when:	 ansible_os_family ==	"RedHat"
- name: Install	the nginx packages
  apt: name={{	item }} state=present	update_cache=yes
  with_items: ubuntu_pkg
  environment: env
  when: ansible_os_family == "Debian"
\end{lstlisting}
Para testar o novo webserver.yml, basta digitar vagrant up ou 
apenas vagrant provision caso sua máquina virtual ainda esteja 
sendo executada. Execute o provisionamento em seguida para 
notar a diferença entre a primeira e a segunda execução. 
Esta é uma maneira simples de verificar a idempotência 
do playbook: na segunda rodada, o atributo changed deve ser 0.



