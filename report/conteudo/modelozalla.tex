%%%%%%%%%%%%%%%%%%%%%%%%%%%%%%%%%%%%%%%% Exibir código fonte


%%%%%%%%%%%%%%%%%%%%%%%%%%%%%%%%%%%%%%%% Lista com pontos
% \begin{itemize}
%   \item [texto]
%   \item [texto]
% \end{itemize}

%%%%%%%%%%%%%%%%%%%%%%%%%%%%%%%%%%%%%%%% Lista com números
%  \begin{enumerate}
%   \item item1
%   \item item2
% \end{enumerate}


\section{Modelo baseado no IBM Bluemix}

\subsection{Introdução ao Bluemix}

A IBM desenvolveu o IBM Bluemix com a ideia de possibilitar a construção de aplicativos sem se preocupar com a infraestrutura, deixando esta a cargo do Bluemix. Assim, construído com base nos projetos open source mais populares do mundo, o IBM Bluemix é uma plataforma em nuvem que permite desenvolvedores construir e executar os apps e serviços mais modernos em tendência no mercado.

\begin{figure}[!htb]
  \centering
  \includegraphics{estrutura}
  \caption{Estrutura do Bluemix}
  \label{Rotulo}
\end{figure}

O Bluemix também é o mais recente produto de nuvem da IBM. Ele permite às organizações e aos desenvolvedores uma maneira rápida e fácil de criar, realizar o \textit{deploy} e gerenciar aplicações na nuvem. Ele é uma implementação da IBM \textit{Open Cloud Architecture} baseada no \textit{Cloud Foundry}, uma plataforma como serviço (mais conhecido como \textit{PaaS – Platform as a Service}). Um de seus maiores diferenciais é entregar serviços de nível empresarial que permitem integrar facilmente as aplicações em nuvem sem precisar se preocupar com questões de instalação e configuração das mesmas, otimizando dimensões de custo e produtividade.

\subsection{Devops e Bluemix}
É o serviço premium de DevOps da Plataforma de nuvem da IBM, que promove um mecanismo de adoção sem fricção e incremental dos serviços de DevOps para o Bluemix, tornando as fases de planejamento ágil e integrado, codificação, \textit{Build}ing e \textit{deploying} mais fáceis.

Enquanto experiência integrada para o desenvolvedor, a plataforma oferece:
\begin{itemize}
    \item Solução na nuvem para desenvolver aplicações
    \item Integra gerenciamento de tarefas, planejamento ágil, controle de recursos
    \item Possibilidade de uso tanto das ferramentas pessoais do desenvolvedor ou da Web IDE
    \item Escalável, seguro e pronto para aplicações empresariais: roda na infraestrutura do \textit{SoftLayer}
\end{itemize}

A ideia geral é que o Bluemix traga as ferramentas automatizadas de infraestrutura e o desenvolvedor se preocupe exclusivamente com o código.

\subsection{Serviços de DevOps disponíveis}

Enquanto principais funcionalidades que fazem do Bluemix o serviço de DevOps na plataforma de nuvem da IBM, são explicitados como recursos de DevOps disponíveis na plataforma:
\begin{itemize}
    \item Acesso fácil: com repositório Git e a Web IDE que já vem construída, é possível começar a codificação dos apps e serviços instantaneamente
    \item Versatilidade de ferramentas: além da Web IDE que é embutida, é possível usar o Eclipse, Visual Studio ou a ferramenta de escolha do desenvolvedor
    \item \textit{Build} & \textit{deploy}: automaticamente construir e realizar o \textit{deploy} das aplicações para o Bluemix
    \item Trabalho em equipe: compartilhar o trabalho e colaborar através de ferramentas especializadas para o desenvolvimento ágil
\end{itemize}

Por exemplo, em um projeto público ou privado no Bluemix, é possível facilmente convidar novos membros ao time, acessar o código de qualquer lugar, construir colaborativamente desde o início, escolher quem vê o projeto e como ele se engaja com comunidades externas.

Nos projetos públicos, é fácil ter acesso e compartilhar trabalho com uma audiência mais ampla. Nos privados, somente é possível compartilhar com a equipe do projeto.

O desenvolvimento ágil na nuvem também é facilitado com os serviços de DevOps do Bluemix por funcionalidades como:
\begin{itemize}
    \item Suporte aos processos do desenvolvimento ágil embutido
    \item Ítens de trabalho para planejar e gerenciar atividades dos projetos
    \item Ferramentas ágeis para o \textit{backlog} do produto, releases e sprints
    \item Dashboard de gráficos para o status do projeto
\end{itemize}

Além disso, é possível escolher onde será feito o desenvolvimento dos serviços:
\begin{itemize}
  \item \textit{Browser}, através da IDE embutida no próprio Bluemix
  \item Desenvolvimento local, através da integração com o Eclipse ou Visual Studio
  \item \textit{Jazz Source Control}, que também tem suporte embutido no Bluemix
  \item Repositório git
  \item GitHub
\end{itemize}
A experiência de DevOps Bluemix, portanto, pode ser sumarizada através do seguinte diagrama:
\begin{figure}[!htb]
    \centering
    \includegraphics{roadmap}
    \caption{\textit{Roadmap} da experiência de DevOps no Bluemix}
    \label{Rotulo}
\end{figure}

\subsection{Serviços}

O conceito de serviços no Bluemix possui um caráter interessante e peculiar. Ele fornece serviços que podem ser usados pelas aplicações sem a necessidade de gerenciar a configuração e operação desse serviços.

\begin{figure}[!htb]
    \centering
    \includegraphics{servico}
    \caption{Tela para adicionar ou ligar serviço no Bluemix}
    \label{Rotulo}
\end{figure}

Os serviços disponíveis são listados em um catálogo na Web UI e também podem ser obtidos usando o comando:
\begin{lstlisting}
cf marketplace
\end{lstlisting}

Para ligar um serviço a uma aplicação que desejar usá-lo, o comando é:
\begin{lstlisting}
cf bs
\end{lstlisting}

Depois de ligar o serviço à aplicação, o Bluemix adicionará detalhes sobre o serviço a uma variável do ambiente que será parseada pelas aplicações.

Um fato interessante no contexto dos serviços no Bluemix é que além de consumir serviços, também é possível criar novos. Enquanto os serviços privados são disponíveis exclusivamente para a organização do usuário, os públicos podem ser adicionados na IBM \textit{Cloud Marketplace} e se tornar um recurso adicional de receita, o que pode ser feito através do estreitamento de relações com a IBM. Para isso, basta que os serviços rodem ou sejam implantados na \textit{SoftLayer} (plataforma de nuvem da IBM), ou se integrem com um dos serviços de plataforma premium da IBM.

\subsection{Características}

Para efetivamente entregar o valor prometido, o Bluemix foi concebido de forma a preservar pontos que estejam adequados aos desafios de negócio das empresas e às expectativas dos desenvolvedores. Assim, se destacam como pontos fortes inerentes ao desenvolvimento e ao uso da plataforma:
\begin{itemize}
    \item \textit{Browser}, através da IDE embutida no próprio Bluemix
    \item  Design centrado no usuário
    \item  Princípios e práticas alinhadas às metodologias ágeis
    \item  Processos e ferramentas de DevOps
    \item  Arquitetura para a nuvem
\end{itemize}

Nesse contexto, o Bluemix possui um forte viés para a importância do desenvolvimento dirigido a testes (não há código que seja escrito se não for para passar em um teste). Os serviços de DevOps no Bluemix podem ser configurados para atumaticamente rodar testes e realizar o \textit{deploy} no Bluemix se o código passar nos testes, obedecendo a um pipeline de DevOps.
\begin{figure}[!htb]
    \centering
    \includegraphics{pipeline}
    \caption{\textit{Pipeline} no Bluemix}
    \label{Rotulo}
\end{figure}

Assim, são benefícios diferenciados inerentes às funcionalidades do Bluemix:
\begin{itemize}
    \item	Contruir e rodar apps: uso de poderosas tecnologias open source em tempo de execuçãoo, de containers e máquinas virtuais para potencializar os apps e serviços
    \item	Acessar dados e aplicações de qualquer lugar: transformação de dados não minerados em informação de valor para o consumo e uso em aplicações de produção
    \item	Usar modelos de \textit{deployment} híbridos e flexíveis: o Bluemix se apresenta como uma plataforma única para várias necessidades, consistente entre entre nuvens públicas, dedicadas e baseadas em condições. É possível ser iniciado de qualquer lugar e facilmente expandir a estratégia com o passar do tempo
\end{itemize}

\subsection{Requisitos Mínimos}

Para o sucesso máximo na execução do Bluemix, são requisitos mínimos do software de \textit{Browser}:
\begin{itemize}
    \item	\textit{Chrome}: versão atualizada do sistema operacional
    \item	\textit{Firefox}: versão atualizada do sistema operacional e ESR 38
    \item	\textit{Internet Explorer}: versão 10 ou 11
    \item	\textit{Safari}: versão mais atualizada do \textit{Browser}
    \item	Interface da linha de comando do \textit{Cloud Foundry}: versão 6.5.1 ou posterior
\end{itemize}

\subsection{Desenvolvimento e gerenciamento de app em Java}

O Bluemix permite diversas possibilidades de desenvolvimento de aplicações. Como proposta de estudo e para fins de normalização de uma instância de projeto, visando comparar com outras alternativas de serviços de DevOps, conforme objetivo do trabalho, serão desenvolvidos no contexto do Bluemix, com documentação e estudo de métricas nos pontos aplicáveis, os seguintes trabalhos:
\begin{enumerate}
    \item Criação de um app Java usando o IBM Bluemix e seus serviços de DevOps
    \item Métricas para localizar e realizar o \textit{fork} de um app java
    \item Configurar \textit{Builds} automáticos e \textit{deploys} pelo Bluemix
    \item Editar um app Java pelo Eclipse e configurar um trigger para \textit{deploy} no Bluemix
    \item Vulnerabilidades em caso de falhas do desenvolvedor
    \item Convidar outros usuários para ajudar no desenvolvimento do app
    \item Gerenciamento do desenvolvimento do app
    \item Planejamento dos sprints no app
\end{enumerate}
