\section{Redes Neurais}

As redes neurais são ferramentas de aprendizado de máquina largamente utilizadas em  problemas de classificação. Haykin \cite{haykin2009neural} define:
\begin{quote}
Uma rede neural é um processador maciçamente paralelamente distribuído constituído de unidade de processamento simples, que têm a propensão natural para armazenar conhecimento experimental e torná-lo disponível para o uso. Ela se assemelha ao cérebro em dois aspectos:

\begin{enumerate}
\item O conhecimento é adquirido pela rede a partir de seu ambiente através de um processo de aprendizagem.
\item Forças de conexão entre neurônios, conhecidos como pesos sinápticos, são utilizados para armazenar o conhecimento adquirido.
\end{enumerate}

\end{quote}

\subsection{Benefícios das redes neurais}

As redes neurais são um recurso computacional com poder de generalização e de estrutura maciçamente paralela. Por esse motivo, elas podem ser implementadas por meio de núcleos integrados que realizam operações simples. 

As redes neurais apresentamas seguintes propriedades\cite{haykin2009neural}:

\begin{enumerate}
\item \emph{Não linearidade.} Um neurônio artificial pode ser linear ou não-linear
\end{enumerate}

\begin{center}
\begin{tabular}{ | c | c | c | }
  \hline
  symbol & value & unit \\ \hline            
  $z Na$ & 11 & - \\ \hline      
  $z F$ & 9 & - \\ \hline      
  $Emax Na$ & 0.545 & $[MeV]$ \\ \hline
\end{tabular}
\end{center}