\chapter{Metodologia}

Inicialmente foi realizado um levantamento bibliográfico a procura de boas referências e ferramentas estabelecidas no contexto da metodologia DevOps. A metodologia DevOps é uma reação à interdependência entre desenvolvimento de software e operações de TI. Pretende ajudar organizações a produzir software e serviços rapidamente. Desde que associações de profissionais e blogs estão tratando do tema somente desde 2009 não existem bibliografias clássicas ou ferramentas estabelecidas. Desse modo optamos pela montagem de três plataformas DevOps: duas plataformas customizadas com ferramentas de código aberto e uma arquitetura proprietária. O intuito é que no fim do trabalho possamos estabelecer métricas a fim de comparar o desempenho das três plataformas.
O primeiro passo para iniciar a montagem das plataformas é a configuração do ambiente onde rodarão as ferramentas das plataformas.
Como o foco do trabalho não é no desenvolvimento da aplicação em si, mas sim nas plataformas de integração entre o time de desenvolvimento e o time de operações utilizaremos aplicações tão somente para estabelecimento de métricas de usabilidade entre as plataformas. Sendo assim, após a configuração do ambiente o seguinte passo será a realização do Build do projeto da aplicação.
Com o Build realizado, o próximo passo é o estabelecimento de testes automatizados de forma que se possa garantir que a aplicação/funcionalidade que irá para a produção conta com um alto grau de confiabilidade.
No próximo passo, o Deploy, contamos finalmente com a aplicação em produção. Restam-se agora o estabelecimento das métricas de monitoração e a análise entre as arquiteturas.
