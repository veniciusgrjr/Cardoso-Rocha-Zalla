\chapter{Ferramentas}

<<<<<<< HEAD
    \section{Banco de Dados}

	\subsection{MySQL} é um sistema de gerenciamento de banco de dados (SGBD), 
	que utiliza a linguagem SQL (Linguagem de Consulta Estruturada) como 
	interface. É atualmente um dos bancos mais populares com mais de 10 
	milhões de instalações pelo mundo.

    \section{Gerenciamento de configurações de software - SCM}

	\subsection{Git} é um sistema de controle de versão distribuído e um 
	sistema de gerenciamento de código fonte com ênfase em velocidade. 
	Cada diretório de trabalho do Git é um repositório com um histórico 
	completo e habilidade total de acompanhamento de revisões, não 
	dependente de acesso a uma rede ou a um servidor central.

	\subsection{GitHub} é um serviço de Web Hosting compartilhado 
	para projetos que usam o controle de versionamento Git. 
	Esse oferece todas as funcionalidades do sistema de controle de 
	revisão e gerenciamento de código (SCM) Git com algumas 
	funcionalidades adicionais.

    \section{Build}

	\subsection{Maven} Apache Maven, ou simplesmente Maven, é uma 
	ferramenta de automação de compilação utilizada primariamente 
	em projetos Java. O Maven utiliza um arquivo XML para descrever 
	o projeto de software sendo construído, suas dependências sobre 
	módulos e componentes externos, a ordem de compilação, 
	diretórios e plug-ins necessários.

    \section{Entrega Contínua - CI}
	\subsection{Jenkins} Jenkins é uma ferramenta open source de 
	integração contínua escrita em Java. Jenkins conta com serviços 
	de integração contínua para desenvolvimento de software. 
	É um sistema de arquitetura servidor rodando em um container 
	servlet tal qual Apache Tomcat. Esse suporta ferramentas SCM 
	incluindo Git e pode ser executado em projetos utilizando 
	Apache Ant a Apache Maven.

    \section{Deployment}
	\subsection{SSH}Parte da suíte de protocolos TCP/IP que torna 
	segura a administração remota de servidores do tipo Unix. O SSH 
	possui as mesmas funcionalidades do TELNET com vantagem da 
	criptografia na conexão entre cliente e o servidor.

    \section{Provisionamento}

	\subsection{Puppet} Utilitário para gerenciamento de configuração de 
	código livre que roda em muitos sistemas Unix compatíveis bem como 
	em Microsoft Windows. Inclui sua própria linguagem declarativa para 
	descrever a configuração do sistema.

	\subsection{Ansible} Plataforma de software livre para configuração e 
	gerenciamento de computadores, combina deployment de software multi 
	nós, execução de tarefas ad hoc e gerenciamento de configurações.
	
	\subsection{Vagrant} 

    \section{Monitoramento}

	\subsection{Nagios} Aplicação de monitoramento de rede de código 
	aberto distribuída sob a licença GPL. Pode monitorar tanto hosts 
	quanto serviços, alertando quando ocorrerem problemas e também 
	quando os problemas são resolvidos.

    \section{Serviços Cloud}

	\subsection{Amazon Web Services} Plataforma de serviços em nuvem 
	segura oferecendo poder computacional, armazenamento de banco de 
	dados, distribuição de conteúdo e outras funcionalidades para 
	ajudar as empresas em seu dimensionamento e crescimento.
	
\section{Containers}
É um novo modelo de virtualização que trabalha no nível de sistema operacional, 
ou seja, ao contrário da máquina virtual, um container não tem visão de uma 
máquina inteira, ele é apenas um processo em execução em um kernel compartilhado 
entre todos os outros containers.

Eles utilizam o namespace para prover o devido isolamento de memória RAM, 
processamento, disco e acesso a rede, ou seja, mesmo compartilhamento o 
mesmo kernel, esse processo em execução tem a visão de estar usando um 
sistema operacional dedicado.

Assim, os containers são leves por não consumir muitos recursos do 
sistema uma vez que por usar o mesmo kernel eles podem executar com 
mais eficiência que uma máquina virtual. Enquanto objetos de software 
virtuais que incluem todos os elementos que um app precisa para executar, 
o container tem benefícios de isolamento e alocação de recurso, sendo 
mais portátil e eficiente que uma máquina virtual. Dessa forma, eles 
ajudam a construir aplicativos de alta qualidade mais rapidamente.

\section{Docker}
Como solução inovadora, o Docker traz diversos serviços e novas 
facilidades que deixam esse modelo muito mais atrativo. Um deles 
é a criação do conceito de “imagens”, que podem ser descritas como 
definições estáticas de como os containers devem ser no momento da 
sua inicialização. São como fotografias de um ambiente. Uma vez 
instanciadas, colocadas em execução, elas assumem a função de 
containers, ou seja, saem da abstração de definição e se transformam 
em processos em execução, dentro de um contexto isolado, que 
enxergam um sistema operacional dedicado pra sí, mas na verdade 
compartilham o mesmo kernel.

Junto a facilidade de uso dos containers, o Docker agregou o 
conceito de nuvem, que dispõe de serviço um para carregar e 
“baixar” imagens Dockers, ou seja, se trata de uma aplicação 
web que disponibiliza um repositório de ambientes prontos, 
onde viabilizou um alto nível de compartilhamento de ambientes.

Com o uso do serviço de nuvem do Docker, percebe-se que a 
adoção do modelo de containers ultrapassa a questão técnica 
e adentra nos assuntos de processo, gerência e atualização 
do ambiente, tornando possível compartilhar facilmente as 
mudanças e viabilizar uma gestão centralizada das definições 
de ambiente.

Utilizando a nuvem Docker, é possível disponibilizar ambientes de 
teste mais leves, acelerando drasticamente o potencial de velocidade 
com que problemas em ambientes integrados são resolvidos.

Assim, tem-se o Docker como um importante projeto open-source 
que automatiza o deploy de aplicações dentro de containers de 
softwares por meio de uma camada adicional de abstração e automação 
em nível da virtualização do sistema operacional. O uso de 
características de isolação de recursos do kernel do Linux 
(cgroups e namespaces do kernel) permitem a containers 
independentes rodar dentro de uma instância particular do Linux, 
evitando o overhead de iniciar e gerenciar máquinas virtuais.
