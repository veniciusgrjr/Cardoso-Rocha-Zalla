\chapter{Ferramentas}

\section{Banco de Dados}

\subsection{MySQL} 

%é um sistema de gerenciamento de banco de dados (SGBD), que utiliza a linguagem SQL (Linguagem de Consulta Estruturada) como interface. É atualmente um dos bancos mais populares com mais de 10 milhões de instalações pelo mundo.

\section{Gerenciamento de configurações de software - SCM}

\begin{itemize}

\item \textbf{Git:} é um sistema de controle de versão distribuído e um sistema de gerenciamento de código fonte com ênfase em velocidade. Cada diretório de trabalho do Git é um repositório com um histórico completo e habilidade total de acompanhamento de revisões, não dependente de acesso a uma rede ou a um servidor central.

\item \textbf{GitHub:} é um serviço de Web Hosting compartilhado para projetos que usam o controle de versionamento Git. Esse oferece todas as funcionalidades do sistema de controle de revisão e gerenciamento de código (SCM) Git com algumas funcionalidades adicionais.


\end{itemize}

\subsection{Build}

\textbf{Maven:} Apache Maven, ou simplesmente Maven, é uma ferramenta de automação de compilação utilizada primariamente em projetos Java. O Maven utiliza um arquivo XML para descrever o projeto de software sendo construído, suas dependências sobre módulos e componentes externos, a ordem de compilação, diretórios e plug-ins necessários.

\subsection{Entrega Contínua - CI}
\textbf{Jenkins:}Jenkins é uma ferramenta open source de integração contínua escrita em Java. Jenkins conta com serviços de integração contínua para desenvolvimento de software. É um sistema de arquitetura servidor rodando em um container servlet tal qual Apache Tomcat. Esse suporta ferramentas SCM incluindo Git e pode ser executado em projetos utilizando Apache Ant a Apache Maven.

\subsection{Deployment}
\textbf{SSH:}Parte da suíte de protocolos TCP/IP que torna segura a administração remota de servidores do tipo Unix. O SSH possui as mesmas funcionalidades do TELNET com vantagem da criptografia na conexão entre cliente e o servidor.

\subsection{Provisionamento}

\begin{itemize}

\item \textbf{Puppet:} utilitário para gerenciamento de configuração de código livre que roda em muitos sistemas Unix compatíveis bem como em Microsoft Windows. Inclui sua própria linguagem declarativa para descrever a configuração do sistema.

\item \textbf{Ansible:} plataforma de software livre para configuração e gerenciamento de computadores, combina deployment de software multi nós, execução de tarefas ad hoc e gerenciamento de configurações.

\end{itemize}

\subsection{Monitoramento}

\textbf{Nagios:}aplicação de monitoramento de rede de código aberto distribuída sob a licença GPL. Pode monitorar tanto hosts quanto serviços, alertando quando ocorrerem problemas e também quando os problemas são resolvidos.

\subsection{Serviços Cloud}

\textbf{Amazon Web Services:}plataforma de serviços em nuvem segura oferecendo poder computacional, armazenamento de banco de dados, distribuição de conteúdo e outras funcionalidades para ajudar as empresas em seu dimensionamento e crescimento.


